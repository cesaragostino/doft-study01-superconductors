\documentclass[11pt]{article}
\usepackage[utf8]{inputenc}
\usepackage{amsmath, amssymb}
\usepackage{graphicx}
\usepackage{hyperref}

\title{DOFT-Inspired Locking Grammar for Superconductors and Superfluids}
\author{Author Name}
\date{}

\begin{document}

\maketitle

\begin{abstract}
We study whether a tightly constrained, DOFT-inspired parametrization can capture quantitative patterns across classical superconductors, high-pressure hydrides, oxides, Fe-based superconductors and superfluid helium. In this framework, each material is represented by a small integer or rational ``fingerprint'' whose exponents are pre-specified by a resonance-locking grammar, rather than fitted freely to the data. A universal correction law with two parameters, $\Gamma$ and $\eta$, is calibrated first on a set of classical metals under non-negativity constraints. Bootstrap ($N=500$) and leave-one-out influence analysis yield $\Gamma \approx O(10^{-17})$ (effectively null curvature) and a robust positive $\eta \approx 4\times 10^{-5}$, indicating that a single dissipative scale suffices to describe this family. Integer fingerprints are remarkably stable across runs and do not significantly distinguish Type-I from Type-II superconductors, suggesting a shared ``integer-locking'' class. In contrast, high-pressure hydrides separate from classical metals along a specific exponent (\texttt{exp\_d\_7}), with a Mann--Whitney test giving an FDR-corrected $p \approx 10^{-4}$ and a moderate effect size (Cliff's $d \approx -0.30$), which provides a quantitative signature for the high-pressure class. In the rational regime, superfluid helium exhibits significantly lower $q$ values ($\approx 2$--$2.3$) than oxides and Fe-based superconductors (Kolmogorov--Smirnov $p \ll 0.01$), while high-pressure systems cluster around $q \approx 5.85$ with tight confidence intervals. A sub-network protocol finds negligible cluster-coupling $\kappa$ for all systems except MgB$_2$, and residual diagnostics show means close to zero with reasonable dispersion, especially for high-pressure materials. Overall, the data are consistent with the idea that a small DOFT-inspired parameter set, combined with a discrete locking grammar, can organize diverse superconducting and superfluid systems without per-material tuning of exponents.
\end{abstract}

\section{Introduction}

Superconductors and superfluids display a wide range of critical temperatures, gap structures and transport behaviors, yet many empirical regularities survive across families. Classical Type-I and Type-II superconductors, high-pressure hydrides, oxide and Fe-based compounds, and superfluid helium all exhibit sharp transitions and non-trivial scaling laws that are not easily reduced to a single microscopic mechanism. A natural question is whether there exists a coarse-grained description that can organize these systems into a small number of classes, using only a few parameters and without per-material tuning of exponents.

The DOFT (Differential Oscillator Field Theory) framework proposes such a coarse-grained description. In DOFT, clusters and ``grainy'' structures are encoded through resonance locks between oscillators, leading to a discrete set of admissible exponents. Rather than fitting continuous exponents freely, the theory prescribes a small library of integer and rational exponents arising from normal locking conditions. In this sense, DOFT is not used here as a microscopic theory of superconductivity, but as a locking grammar that constrains how effective fingerprints are built from a limited set of exponents.

In this work we take a deliberately conservative stance: we do not attempt to prove DOFT as a fundamental theory. Instead, we ask a narrower and more practical question: given a DOFT-inspired locking grammar and a minimal correction law, can we describe a heterogeneous collection of superconductors and superfluid helium with a small, fixed set of parameters? Concretely, each material is mapped to either an integer fingerprint or a rational fingerprint $q$, whose exponents are fixed by the locking rules and not adjusted per material. A universal correction law with parameters $\Gamma$ and $\eta$ is then calibrated on classical metals under weak constraints ($\Gamma \ge 0$, $\eta \ge 0$) and reused across families.

Using this setup, we perform a series of statistical tests on published data for classical superconductors, high-pressure hydrides, oxides, Fe-based superconductors and superfluid helium. We first show that the calibration on classical metals yields $\Gamma \approx 0$ and a robust positive $\eta$, as confirmed by bootstrap and leave-one-out influence analysis. We then examine the stability of integer fingerprints across runs and their ability (or lack thereof) to distinguish Type-I from Type-II superconductors. Non-parametric tests (Kruskal--Wallis, Mann--Whitney, Kolmogorov--Smirnov) and multiple-comparison corrections are used to quantify differences between families in specific exponents and in the rational fingerprint $q$. Finally, we explore a sub-network protocol for detecting cluster-level couplings, and we assess residual diagnostics by family to test whether the proposed correction law removes systematic drift.

Our results indicate that (i) a single dissipative scale $\eta$ suffices to describe classical metals, with $\Gamma$ effectively null; (ii) Type-I and Type-II superconductors share a common integer fingerprint within the noise of our data; (iii) high-pressure hydrides separate from classical metals along a specific integer exponent, yielding a moderate but statistically robust effect; (iv) superfluid helium and high-pressure hydrides occupy distinct, tightly clustered regions in rational $q$-space, clearly separated from oxides and Fe-based superconductors; and (v) within the current protocol, cluster-level couplings are not required to explain the data, except for a small but detectable effect in MgB$_2$. We interpret these findings as evidence that a DOFT-inspired locking grammar, combined with a minimal correction law, can serve as a compact organizational scheme for diverse superconducting and superfluid systems, while remaining fully testable and falsifiable as new data become available.

\end{document}
