\documentclass[aps,prb,reprint,superscriptaddress,nofootinbib,floatfix]{revtex4-2}

% Encoding and fonts
\usepackage[utf8]{inputenc}
\usepackage[T1]{fontenc}

% Math and symbols
\usepackage{amsmath,amssymb,amsfonts,mathtools}
\usepackage{siunitx}

% Figures, tables, hyperlinks
\usepackage{graphicx}
\graphicspath{{figures/}}
\usepackage{booktabs}
\usepackage{hyperref}
\usepackage[dvipsnames]{xcolor}

\hypersetup{
  colorlinks=true,
  linkcolor=MidnightBlue,
  citecolor=MidnightBlue,
  urlcolor=MidnightBlue
}

% Convenience macros
\newcommand{\DOFT}{\textsc{doft}}
\newcommand{\Tc}{T_{\mathrm{c}}}
\newcommand{\EF}{E_{\mathrm{F}}}
\newcommand{\Debye}{\Theta_{\mathrm{D}}}
\newcommand{\GammaDOFT}{\Gamma}
\newcommand{\etaDOFT}{\eta}
\newcommand{\qDOFT}{q}
\newcommand{\Robs}{R_{\mathrm{obs}}}
\newcommand{\Rcorr}{R_{\mathrm{corr}}}
\newcommand{\logres}{\ell_{\eta}} % log_residual_eta

\begin{document}

\title{Prime-space fingerprints of superconductors and superfluid helium\\
from a delayed-oscillator locking grammar}

\author{C.~Agostino}
\email{[cesar.agostino@gmail.com]}
\affiliation{Independent Researcher, Buenos Aires, Argentina}
% \affiliation{[Add a formal affiliation here if appropriate]}

\date{\today}

\begin{abstract}
Delayed-oscillator field theory (\DOFT) is a speculative framework in which
matter and geometry emerge from a dense network of coupled oscillators with
finite memory. In this work we do not attempt to justify \DOFT{} as a
fundamental description of Nature. Instead we ask a narrower, more practical
question: given a \DOFT{}-inspired locking grammar and a minimal correction
law, can we describe a heterogeneous collection of superconductors and
superfluid helium using a small, fixed set of parameters?

Each material is mapped to an “integer fingerprint’’ or a “rational
fingerprint’’ whose exponents are fixed by the locking rules and not tuned
per material. A universal correction law with parameters
$(\GammaDOFT,\etaDOFT)$ is calibrated once on classical superconducting
elements and then reused across families. On an extended dataset including
binary, molecular, iron-based and high-pressure superconductors, as well as
superfluid $^4$He, we find that the same $(\GammaDOFT,\etaDOFT)$ values
produce logarithmic residuals with mean $\approx 0$ and modest variance
within most families. Integer fingerprints show stable prime exponents by
family, while rational fingerprints yield narrow, family-specific
distributions of denominators $\qDOFT$ in the range $q\sim 2$--$8$.
High-pressure superconductors and superfluid helium separate cleanly in
$\qDOFT$, providing a quantitative “prime-space signature’’ of their
macroscopic behaviour within this grammar. A weak next-neighbour coupling
parameter $\kappa$ produces only small, localized improvements concentrated
in MgB$_2$, and leaves the global fingerprints essentially unchanged.

We interpret these patterns as empirical constraints on any delayed-oscillator
model that attempts to use a small set of locks to organize the superconducting
landscape, rather than as proof of \DOFT{} itself. All code and data required
to reproduce the analysis are provided in an open repository.
\end{abstract}

\maketitle

%------------------------------------------------------------
\section{Introduction}\label{sec:intro}

The delayed-oscillator field theory (\DOFT) program starts from a simple but
aggressive hypothesis: at sufficiently small scales the relevant degrees of
freedom are not point particles on a fixed spacetime, but a dense network of
coupled oscillators with finite-range delays and memory kernels. In this
picture, what we call “space’’ and “time’’ emerge from patterns of coherence
and decoherence inside that network, and material properties reflect how
different collectives lock into a small set of preferred frequency ratios.

In this work we take a deliberately conservative stance: we do not attempt to
prove \DOFT{} as a fundamental theory. Instead, we ask a narrower and more
practical question: \emph{given a \DOFT{}-inspired locking grammar and a
minimal correction law, can we describe a heterogeneous collection of
superconductors and superfluid helium with a small, fixed set of
parameters?} Concretely, each material is mapped to either an integer
fingerprint or a rational fingerprint $\qDOFT$, whose exponents are fixed by
the locking rules and not adjusted per material. A universal correction law
with parameters $(\GammaDOFT,\etaDOFT)$ is then calibrated on classical
metals under weak constraints $(\GammaDOFT \ge 0, \etaDOFT \ge 0)$ and reused
across families.

From the point of view of conventional condensed-matter theory, this is an
unusual modelling choice. Standard approaches start from microscopics
(electronic structure, phonon spectra, pairing mechanisms) and build upwards.
Here we go the other way: we start from a simple locking grammar in
prime-space and ask whether it can describe a curated cross-section of the
superconducting landscape without material-by-material tuning. The resulting
fingerprints are not proposed as replacements for microscopic theories, but
as a coarse, “spectral’’ summary of how different macroscopic families sit
relative to the same discrete lattice of ratios.

%Optionally: a brief paragraph connecting to delays/memory literature.
Delayed and memoryful dynamics are ubiquitous in climate models,
physiological control and nonequilibrium statistical mechanics\cite{BCS1957,
Tinkham1996,Efron1993,Mori1965,Zwanzig2001}.  The\DOFT{} program borrows 
mathematical tools from that literature but applies them in a different 
regime: as organizing principles rather than detailed time-domain predictions.

%------------------------------------------------------------
\section{Methods}\label{sec:methods}

\subsection{Dataset and macroscopic jumps}

Our dataset aggregates superconductors and superfluid helium into a common
format. Each row corresponds to a material and a macroscopic “sub-network’’
(single band, $\sigma$, $\pi$, or specific vibrational modes under pressure),
and includes:
critical temperature $\Tc$, superconducting gap $\Delta$, Debye temperature
$\Debye$, and Fermi energy $\EF$ when available. From these we construct a
small set of dimensionless ratios $R$ (“jumps’’) such as
$\Tc\to\Delta$, $\Delta\to\Debye$, $\Debye\to\EF$, and inter-band
$\sigma$–$\pi$ jumps for MgB$_2$ and related systems.

The v6 dataset used here extends a previous run by adding $\mathcal{O}(50)$
additional superconductors, especially in the iron-based and high-pressure
families. Classical superconducting elements are used exclusively for
calibrating $(\GammaDOFT,\etaDOFT)$; all other families are held out for
testing.

\subsection{DOFT-inspired locking grammar}

Each macroscopic jump is assigned to a “lock’’ on a discrete lattice generated
by the first primes $\{2,3,5,7\}$. In the integer case we write
%
\begin{equation}
  R_{\mathrm{lock}} = 2^{a} 3^{b} 5^{c} 7^{d},
\end{equation}
%
with nonnegative integer exponents $(a,b,c,d)$ constrained by a small set of
rules (e.g.\ maximum total exponent, allowed patterns per family). In the
rational case we further write
%
\begin{equation}
  R_{\mathrm{lock}} = p/q,
\end{equation}
%
with $p,q$ co-prime and $q$ restricted to a small range ($q \le 8$ in the
present study). For each jump we choose the lock that minimizes the
residual under the correction law described below.

The resulting exponent vectors $(a,b,c,d)$ define the
\emph{integer fingerprint} of a family, while the distribution of $q$
defines its \emph{rational fingerprint}. We refer to these collectively as
“prime-space fingerprints’’.

\subsection{Universal correction law and calibration}\label{subsec:calibration}

To account for systematic deviations between the ideal locks and the observed
ratios, we introduce a simple correction law parametrized by
$\GammaDOFT$ and $\etaDOFT$. For each observed jump $\Robs$ and chosen lock
$R_{\mathrm{lock}}$ we define a corrected value $\Rcorr$ and a
logarithmic residual
%
\begin{equation}
  \logres = \log_{10}\left( \frac{\Robs}{\Rcorr} \right),
\end{equation}
%
constructed so that $\logres = 0$ corresponds to perfect agreement with the
ideal lock plus universal correction.

The parameters $(\GammaDOFT,\etaDOFT)$ are calibrated using only classical
superconducting elements, with bootstrap resampling and
leave-one-out (LOO) influence diagnostics to quantify uncertainty and
stability. In the baseline configuration ($w=800$, prime cutoff $p=7919$) we
obtain
%
\begin{equation}
  \GammaDOFT \sim \mathcal{O}(10^{-17}), \qquad
  \etaDOFT \approx 4\times 10^{-5},
\end{equation}
%
with narrow bootstrap confidence intervals and modest LOO sensitivity.
These values are then \emph{frozen} and reused for all other families.

\begin{figure*}[t]
  \centering
  \includegraphics[width=\textwidth]{fig01_calibration}
  \caption{%
    Calibration of the universal correction parameters
    $(\GammaDOFT,\etaDOFT)$ on classical superconducting elements.
    Left: bootstrap distribution of $\etaDOFT$.
    Middle: bootstrap distribution of $\GammaDOFT$.
    Right: leave-one-out (LOO) influence of each element on $\etaDOFT$,
    expressed as a percentage shift relative to the full fit.}\label{fig:calibration}
\end{figure*}

\subsection{Cluster coupling parameter \texorpdfstring{$\kappa$}{kappa}}

We also explore a weak coupling parameter $\kappa$ designed to capture
inter-jump correlations inside a material or sub-network. Setting
$\kappa=0$ corresponds to treating each jump independently under the
universal correction law, while $\kappa>0$ allows small shifts in the
effective locks driven by neighbouring jumps. In practice, we find that
nonzero $\kappa$ only produces noticeable changes in MgB$_2$, and leaves both
the global residual statistics and the prime-space fingerprints almost
unchanged.

%------------------------------------------------------------
\section{Results}\label{sec:results}

\subsection{Integer fingerprints by family}

Using the baseline configuration $(w=800,p=7919)$ with $\kappa=0$, we obtain
integer fingerprints for each family by aggregating the exponents
$(a,b,c,d)$ across all factorized jumps. Figure~\ref{fig:integer_fp} shows
violin plots of the distributions of $a$ (prime $2$) and $d$ (prime $7$) for
selected families.

\begin{figure*}[t]
  \centering
  \includegraphics[width=\textwidth]{fig02_integer_fingerprint}
  \caption{%
    Integer fingerprints by family. Left: distribution of exponents on prime
    $7$ ($d=\mathrm{exp\_d\_7}$). Right: distribution of exponents on prime
    $2$ ($a=\mathrm{exp\_a\_2}$). Each violin summarizes the factorized jumps
    in that family under the baseline configuration.}\label{fig:integer_fp}
\end{figure*}

Type-I and Type-II superconductors exhibit remarkably stable mean exponent
vectors across bootstrap resamples, with typical values
$(\langle a\rangle,\langle b\rangle,\langle c\rangle,\langle d\rangle)$
close to $(1.5,0.8,0.5,0.4)$ and $(1.9,0.6,0.5,0.4)$ respectively.
High-pressure and iron-based families show distinct patterns, including a
nearly pure $2\times 7$ structure in certain iron-based $\sigma$ subnetworks.
We interpret these as “chords’’ in the prime lattice that characterize each
macroscopic family.

\subsection{Rational fingerprints and separation of families}

The rational fingerprints focus on the denominators $\qDOFT$ of the locks
chosen at the factorization stage. Figure~\ref{fig:rational_q} shows the
distribution of $\qDOFT$ for superfluid helium, high-pressure
superconductors, and iron-based superconductors.

\begin{figure}[t]
  \centering
  \includegraphics[width=\columnwidth]{fig03_rational_q}
  \caption{%
    Rational fingerprints: distribution of denominators $q$ by family
    (superfluid, high-pressure superconductors, iron-based superconductors).
    The superfluid cluster sits at low $q\sim 2$--$2.3$ with a narrow
    spread, while high-pressure superconductors favour $q\sim 3$--$7$ and
    iron-based systems cluster near $q=1$.}\label{fig:rational_q}
\end{figure}

Superfluid $^4$He is clearly distinct: its $\qDOFT$ distribution is narrow
and centred at low values ($q\sim 2$--$2.3$). High-pressure superconductors
occupy a broader band around $q\sim 3$--$7$, while iron-based superconductors
are concentrated near $q=1$. A Mann-Whitney test on $q$ or on selected
exponent components identifies a statistically significant separation between
high-pressure superconductors and the classical/iron-based families, with a
moderate effect size.

\subsection{Residual structure and family-level fingerprints}

Figure~\ref{fig:residuals} summarizes the logarithmic residuals $\logres$ by
family and sub-network. The mean residuals are close to zero in most groups,
with standard deviations at the level expected from the calibration set.

%\begin{figure*}[t]
%  \centering
%  \includegraphics[width=\textwidth]{fig04_residuals}
%  \caption{%
%    Logarithmic residuals by family. Left: mean $\pm$ standard deviation of
%    $\logres$ for each (category, sub-network) pair. Bars in blue indicate
%    groups with $N\ge 5$; orange bars correspond to small-$N$ groups.
%    Right: scatter of $\logres$ versus effective dimension $d$ for
%    high-pressure, iron-based, and superfluid families.}\label{fig:residuals}
%\end{figure*}

\begin{figure}
  \centering
  \includegraphics[width=\columnwidth]{fig04a_residual_means}
  \caption{Mean and standard deviation of the log-residual
  \(\log_{10} r_{\mathrm{obs}}/r_{\mathrm{prime}}\) for each family and
  sub-network. Blue bars correspond to groups with \(N \ge 5\), orange bars
  to groups with \(N < 5\). Most canonical networks (binary, molecular,
  iron-based, and high-pressure single) are tightly centered around zero,
  while superfluid helium and a few high-pressure modes appear systematically
  compressed (negative mean).}
  \label{fig:residual_means}
\end{figure}

\begin{figure}
  \centering
  \includegraphics[width=\columnwidth]{fig04b_residual_vs_d}
  \caption{Scatter of log-residuals \(\log_{10} r_{\mathrm{obs}}/r_{\mathrm{prime}}\)
  versus the effective dimension \(d\) for high-pressure, iron-based, and
  superfluid networks. The spread remains moderate across \(d = 1,2,3\),
  with no runaway tails; most deviations stay within a factor of a few in
  either direction.}
  \label{fig:residual_vs_d}
\end{figure}

Certain groups, notably superfluid helium and a subset of high-pressure
modes (e.g.\ La acoustic $\sigma$, H1 optic $\sigma$), show systematically
negative mean residuals, indicating that the observed ratios sit slightly
below the ideal locks plus universal correction. Others, such as binary
$\pi$-band superconductors and molecular superconductors, show extremely
tight residuals around zero. This supports the interpretation that
$(\GammaDOFT,\etaDOFT)$ act as a genuinely universal correction across
families, with family-specific structure encoded primarily in the prime-space
fingerprints.

\subsection{Effect of the coupling parameter \texorpdfstring{$\kappa$}{kappa}}

Finally, we compare the baseline $\kappa=0$ run to a run with small but
nonzero $\kappa$, using the same $w=800$ and $p=7919$.
Figure~\ref{fig:kappa} shows the difference in post-correction error
($\Delta\mathrm{error} = \mathrm{err\_after\_kappa}-\mathrm{err\_after\_eta}$)
for all jumps in the cluster core and for the most affected cases.

%\begin{figure*}[t]
%  \centering
%  \includegraphics[width=\textwidth]{fig05_kappa_vs_no_kappa}
%  \caption{%
%    Impact of the coupling parameter $\kappa$. Left: change in residual error
%    for all cluster-core jumps when turning on $\kappa$ at the baseline
%    configuration. Right: zoom on the few cases with the largest absolute
%    change. Significant shifts are concentrated in MgB$_2$; most other
%    materials are essentially unaffected.}\label{fig:kappa}
%\end{figure*}

\begin{figure}
  \centering
  \includegraphics[width=\columnwidth]{fig05_kappa_delta_hist}
  \caption{Distribution of the difference in locking error
  \(\Delta \mathrm{err} = \mathrm{err}_{\kappa} - \mathrm{err}_{\eta}\)
  across all cluster-core jumps. Almost all values lie extremely close
  to zero, indicating that the \(\kappa\)-correction has negligible
  effect on the global fit. Only a few jumps in the tails show visible
  deviations.}
  \label{fig:kappa_hist}
\end{figure}

\begin{figure}
  \centering
  \includegraphics[width=\columnwidth]{fig06_kappa_topdelta}
  \caption{Largest \(\kappa\)–0 differences in locking error
  \(|\Delta \mathrm{err}| = |\mathrm{err}_{\kappa} - \mathrm{err}_{\eta}|\)
  across cluster-core jumps (top ten). The dominant contributions all come
  from MgB\(_2\) in its \(\sigma\) and \(\pi\) subnetworks, while only a
  few non--MgB\(_2\) jumps show much smaller shifts. This confirms that
  \(\kappa\) mainly refines a handful of multi-gap, strongly coupled
  cases and leaves the rest of the network essentially unchanged.}
  \label{fig:kappa_top}
\end{figure}

Out of the full set of factorized jumps, only a handful---all in MgB$_2$
$\sigma$ and $\pi$ subnetworks---show noticeable changes in the corrected
ratios or residuals. For the remaining materials, the choice of $\kappa$
within the tested range has negligible effect on both the residual statistics
and the prime-space fingerprints. Within this grammar, there is therefore no
evidence for strong cluster-level couplings beyond the universal correction
already captured by $(\GammaDOFT,\etaDOFT)$.

%------------------------------------------------------------
\section{Discussion and outlook}\label{sec:discussion}

The main outcome of this study is not a new microscopic model of
superconductivity, but an empirical statement: a simple locking grammar on
the primes $\{2,3,5,7\}$, combined with a universal two-parameter correction
law calibrated on classical metals, is sufficient to organize a heterogeneous
set of superconductors and superfluid helium into stable prime-space
fingerprints. These fingerprints are robust under changes in the dataset
(v6 vs.\ earlier runs), under variation of technical parameters
($w$ and prime cutoff), and under the inclusion of a weak coupling
$\kappa$.

From a \DOFT{} perspective, this suggests that if a delayed-oscillator
network underlies these systems, it may operate with a small, discrete set
of preferred ratios and a universal correction that is largely insensitive to
microscopic details. From a more conventional perspective, the fingerprints
can be viewed as a compact, phenomenological summary of how different
families sit relative to the same discrete lattice of scales.

We interpret these patterns as the prime-space signature of each macroscopic
family under the \DOFT{}-inspired grammar. Whether this signature points to a
deeper delayed-oscillator structure, or simply reflects hidden regularities
in the curated dataset, is an open question for future work.

%------------------------------------------------------------
\section*{Acknowledgements}

The author thanks colleagues and online communities for discussions and for
making high-quality superconductivity and superfluid datasets publicly
available. This project made extensive use of open-source software, including
Python, NumPy, SciPy, pandas, Matplotlib, and Jupyter.

AI tools were used as assistants in the development of the code and this
manuscript. In particular, OpenAI GPT-5.1~Thinking, Google Gemini~Pro, and
Anthropic Claude were employed for drafting text, refactoring analysis
pipelines, and checking statistical procedures. All modelling decisions,
dataset curation and interpretations presented here were made and verified by
the human author.

%------------------------------------------------------------
\bibliographystyle{apsrev4-2}
\bibliography{references}

\end{document}
